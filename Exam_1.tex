\documentclass[11pt]{article}
%\documentclass[11pt,twoside]{article}
\usepackage[letterpaper,margin=0.75in,left=1.25in,nohead]{geometry}
\usepackage{graphicx}
\usepackage{color}
\usepackage{amsmath}
\usepackage{amsfonts}
\usepackage{bbm}
\usepackage{tikz}
\usetikzlibrary{intersections}
\usepackage{subcaption}
\usepackage{mathrsfs}
\usepackage{mdframed}
\usepackage{makecell}

\begin{document}
\title{Complex Analysis: Exam 1B}
\author{Kevin Roebuck}
\maketitle

\begin{enumerate}
\item (20) Miscellaneous computations. Show all necessary steps.
  \begin{enumerate}
  \item Compute all values of $(1 - i)^{\frac{4}{3}} = ((1 - i)^4)^{\frac{1}{3}}$.

  \begin{mdframed}[align=left]
    Begin by writing $z = 1 - i$ in polar form. The modulus is $|z| = \sqrt{1^2 + (-1)^2} = \sqrt{2}$. Then, by using the relationships
    \begin{align*}
      \cos\theta = \frac{\text{Re}(z)}{|z|} \qquad \text{and} \qquad \sin\theta = \frac{\text{Im}(z)}{|z|},
    \end{align*}
    we see that $\cos\theta = 1/\sqrt{2}$ and $\sin\theta = -1/\sqrt{2}$. From this set of equations we conclude that Arg$(z) = -\pi/4$. Therefore $z = \sqrt{2}e^{-i\pi/4}$. Raising $z$ to the fourth power gives $z^4 = 4e^{-i\pi}$ = -4. Now all that's left to do is find the cube roots of $z^4$, which will require the use of
    \begin{equation*}
      z^{1/m} = |z|^{1/m}e^{i(\theta + 2k\pi)/m} \qquad (k = 0,1,2,...,m-1).
    \end{equation*}
    In our case $m = 3$ so that the above equation takes the form
    \begin{equation*}
      (z^4)^{1/3} = \left|z^4\right|^{1/3}e^{i(-\pi + 2k\pi)/3}
      = 4^{1/3}e^{i\pi(2k - 1)/3}
      \qquad (k = 0,1,2).
    \end{equation*}
    By plugging in the possible values of $k$, we get:

    \begin{center}
    \begin{tabular}{|l|l|}\hline
      $k$ & $z^{4/3}$ \\\Xhline{2\arrayrulewidth}
      0 & $4^{1/3}e^{-i\pi/3}$ \\\hline
      2 & $4^{1/3}e^{i\pi/3}$ \\\hline
      2 & $4^{1/3}e^{i\pi}$ \\\hline
    \end{tabular}
    \end{center}

    A plot of these roots, along with the vector representing $z^4$, can be seen in the figure below.

    \begin{center}
    \begin{tikzpicture} [scale=1.2]
      \draw[-] (0,0) -- (4,0); %draw axes
      \draw[-] (0,0) -- (0,4);
      \draw[line width=0.5mm, ->] (0,0) -- (-4,0); %draw z^4 vector
      %\draw[opacity=0.2] (0,0) circle(1.587cm); % draw circle
      \draw[line width=0.5mm, ->] (0,0) -- (-1.587,0); %draw cube root vectors
      \draw[line width=0.5mm, ->] (0,0) -- (0.793,-1.375);
      \draw[line width=0.5mm, ->] (0,0) -- (0.793,1.375);

      %draw labels
      \node at (4.2,0) {$x$};
      \node at (0,4.2) {$y$};
      \node at (-4.1,0.2) {$z^4$};
      \node at (1.7,-1.5) {$z^{4/3} (k=0)$};
      \node at (1.7,1.5) {$z^{4/3} (k=1)$};
      \node at (-1,0.3) {$z^{4/3} (k=2)$};
    \end{tikzpicture}
    \end{center}
  \end{mdframed}
  
  \item Compute ${\displaystyle \int_0^{2\pi}} \cos^5(x)\,dx$ using methods introduced in our class.
  
  \begin{mdframed}
    Start by simplifying the integrand using Euler's formula
    \begin{equation*}
      \cos^5 x = \left[\frac{e^{ix} + e^{-ix}}{2}\right]^5 = \frac{1}{2^5}[e^{ix} + e^{-ix}]^5.
    \end{equation*}
    Expanding via the binomial formula gives
    \begin{align*}
      \cos^5 x &= \frac{1}{2^5}[e^{5ix} + 5e^{4ix}e^{-ix} + 10e^{3ix}e^{-2ix} + 10e^{2ix}e^{-3ix} + 5e^{ix}e^{-4ix} + e^{-5ix}] \\
      &= \frac{1}{2^5}[(e^{5ix} + e^{-5ix}) + 5(e^{3ix} + e^{-3ix}) + 10(e^{ix} + e^{-ix})] \\
      &= \frac{1}{2^5}[(2\cos 5x) + (2\cos 3x) + (2\cos x)] \\
      &= \frac{1}{2^4}[\cos 5x + \cos 3x + \cos x].
    \end{align*}
    Therefore
    \begin{align*}
      \int_0^{2\pi}\cos^5 x\, dx &= \frac{1}{2^4}\int_0^{2\pi}\left[\cos 5x + \cos 3x + \cos x\right]dx \\
      &= \frac{1}{2^4}\left[\frac{1}{5}\sin 5x + \frac{5}{3}\sin 3x + 10\sin x\right]_0^{2\pi} \\
      &= 0.
    \end{align*}
  \end{mdframed}

  \item Find the partial fraction decomposition of ${\displaystyle \frac{z^2 + z + 1}{(z - i)^2 (z + 2)}}$. (You do not need to simplify the constants that you solve for.)

  \begin{mdframed}
    The desired form is
    \begin{equation*}
      R(z) \equiv \frac{z^2 + z + 1}{(z - i)^2 (z + 2)} = \frac{A_0^{(1)}}{z+2} + \frac{A_0^{(2)}}{(z - i)^2} + \frac{A_1^{(2)}}{z - i}.
    \end{equation*}
    With the function in this form, we are now able to make use of the general expression for the coefficients of the partial fraction decomposition of a rational function $R_{m,n}(z)$ (whose denominator degree $n = d_1 + d_2 + ... + d_r$ exceeds its numerator degree $m$):
    \begin{equation*}
      A_s^{(j)} = \lim_{z\rightarrow\xi_j}\frac{1}{s!}\frac{d^s}{dz^2}\left[(z - \xi_j)^{d_j}R_{m,n}(z)\right],
    \end{equation*}
    where $\xi_j$ are distinct roots and $d_j$ is the degree of the root. We find that
    \begin{equation*}
      A_0^{(1)} = \lim_{z\rightarrow -2}(z + 2)R(z) = \lim_{z\rightarrow -2} \frac{z^2 + z + 1}{(z - i)^2} = \frac{3}{3 + 4i} = \frac{3}{3 + 4i}\left(\frac{3 - 4i}{3 - 4i}\right) = \frac{9}{25} - \frac{12}{25}i.
    \end{equation*}
    The next coefficient is
    \begin{equation*}
      A_0^{(2)} = \lim_{z\rightarrow i} (z - i)^2 R(z) = \lim_{z\rightarrow i} \frac{z^2 + z + 1}{z + 2} = \frac{i}{i + 2} = \frac{i}{i + 2}\left(\frac{-i + 2}{-i + 2}\right) = \frac{1}{5} + \frac{2}{5}i.
    \end{equation*}
    Finally, we have
    \begin{align*}
      A_1^{(2)} &= \lim_{z\rightarrow i} \frac{d}{dz}\left[(z - i)^2 R(z)\right] = \lim_{z\rightarrow i} \frac{d}{dz}\left[\frac{z^2 + z + 1}{z + 2}\right] = \lim_{z\rightarrow i} \left[\frac{(2z + 1)(z + 2) - (z^2 + z + 1)}{(z + 2)^2}\right] \\
      &= \lim_{z\rightarrow i} \left[\frac{z^2 + 4z + 1}{(z + 2)^2}\right] = \frac{4i}{3 + 4i} = \frac{4i}{3 + 4i}\left(\frac{3 - 4i}{3 - 4i}\right) = \frac{16}{25} + \frac{12}{25}i.
    \end{align*}
    Putting everything together:
    \begin{equation*}
      R(z) \equiv \frac{z^2 + z + 1}{(z - i)^2 (z + 2)} = \frac{\frac{9}{25} - \frac{12}{25}i}{z+2} + \frac{\frac{1}{5} + \frac{2}{5}i}{(z - i)^2} + \frac{\frac{16}{25} + \frac{12}{25}i}{z - i}.
    \end{equation*}
  \end{mdframed}
  
  \item Describe the set of points $|z| = 2|z - i|$.

  \begin{mdframed}
  The modulus of a complex number $z = x + iy$ is given by $|z| = (x^2 + y^2)^{1/2}$. To avoid dealing with the square roots, let us square both sides of the given equation and write out the moduli in terms of $x$ and $y$
  \begin{align*}
    x^2 + y^2 &= 4(x^2 + (y-1)^2) \\
    &= 4(x^2 + y^2 - 2y + 1).
  \end{align*}
  Grouping terms together yields
  \begin{equation*}
    x^2 + \left(y^2 - \frac{8}{3}y\right) = -\frac{4}{3}.
  \end{equation*}
  We shall next complete the square by adding $(4/3)^2$ to each side
  \begin{equation*}
    x^2 + \left(y^2 - \frac{8}{3}y + \frac{16}{9}\right) = -\frac{4}{3} + \frac{16}{9},
  \end{equation*}
  which simplifies to
  \begin{equation*}
    x^2 + \left(y - \frac{4}{3}\right)^2 = \frac{4}{9}.
  \end{equation*}
  This is the equation for a circle centered about the point $(x,y) = (0,4/3)$ and with radius $(4/9)^{1/2} = 2/3$. Therefore, the set of points that satisfy $|z| = 2|z - i|$ are all of the points on the circle centered about the point $(x,y) = (0,4/3)$ and with radius $2/3$.
  \end{mdframed}
  
  \end{enumerate}

\item (5) Use the idea of local linear approximation to describe what happens to a small disc centered at $z_0 = 3 + 4i$ when it is substituted into $f(z) = 1/z$. Describe what happens in the language of translations, rotations, expansions, and/or contractions. Draw an appropriate diagram.

\begin{mdframed}
  
\end{mdframed}


\item (10) Describe the projection on the Riemann Sphere of the set $\{z = x + iy : x \geq \sqrt{3}\}$. I encourage you to include a nice diagram.

\begin{mdframed}
  
\end{mdframed}


\item (15) Choose one of the following explorations.
\begin{enumerate}
  \addtocounter{enumii}{1}
  \item  An exploration of {\it admissibility}. Recall that if $z = x + iy$, then $x = \frac{1}{2}(z + \bar{z})$ and $y = \frac{1}{2i}(z - \bar{z})$.
  
  \begin{enumerate}
  \item (4) Substitute for $x$ and $y$ in $f(x,y) = x^2 - y^2 + i2xy$ and verify that $f$ reduces to a function of $z$ only, i.e. that there are no $z$ terms remaining after simplification. Also use the Cauchy Riemann Equations to verify that $f$ is analytic.

  \begin{mdframed}
  
  \end{mdframed}

    
  \item (4) Substitute for $x$ and $y$ in $g(x,y) = x^2 - y^2 + i 3 x y$ and verify that $f$ does not reduce to a function of $z$ only, i.e. that there are still $z$ terms remaining after simplification. Also use the Cauchy Riemann Equations to verify that $g$ is not analytic.

  \begin{mdframed}
  
  \end{mdframed}

  \item (7) {\it Admissibility} essentially boils down to the idea that $f(x,y) = u(x, y) + iv(x, y)$ does not depend on $z$. A more precise way to say this is that $\frac{\partial f}{\partial \bar{z}} = 0$. Use the multivariable chain rule,
  \begin{equation*}
    \frac{\partial f}{\partial \bar{z}} = \frac{\partial f}{\partial x} \frac{\partial x}{\partial \bar{z}} + \frac{\partial f}{\partial y} \frac{\partial y}{\partial \bar{z}},
  \end{equation*}
  to show that $\frac{\partial f}{\partial \bar{z}} = 0$ if the Cauchy Riemann Equations are satisfied. This indicates that $f$ does not depend on $z$ if $f$ is analytic.

  \begin{mdframed}
  
  \end{mdframed}

  \end{enumerate}
\end{enumerate}

\end{enumerate}

\end{document}